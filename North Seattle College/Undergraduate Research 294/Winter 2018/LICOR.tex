%%%%%%%%%%%%%%%%%%%%%%%%%%%%%%%%%%%%%%%%%
% Journal Article
% LaTeX Template
% Version 1.4 (15/5/16)
%
% This template has been downloaded from:
% http://www.LaTeXTemplates.com
%
% Original author:
% Frits Wenneker (http://www.howtotex.com) with extensive modifications by
% Vel (vel@LaTeXTemplates.com)
%
% License:
% CC BY-NC-SA 3.0 (http://creativecommons.org/licenses/by-nc-sa/3.0/)
%
%%%%%%%%%%%%%%%%%%%%%%%%%%%%%%%%%%%%%%%%%

%----------------------------------------------------------------------------------------
%	PACKAGES AND OTHER DOCUMENT CONFIGURATIONS
%----------------------------------------------------------------------------------------

\documentclass[twoside,twocolumn]{article}

\usepackage{blindtext} % Package to generate dummy text throughout this template 

\usepackage[sc]{mathpazo} % Use the Palatino font
\usepackage[T1]{fontenc} % Use 8-bit encoding that has 256 glyphs
\linespread{1.05} % Line spacing - Palatino needs more space between lines
\usepackage{microtype} % Slightly tweak font spacing for aesthetics

\usepackage[english]{babel} % Language hyphenation and typographical rules

\usepackage[hmarginratio=1:1,top=32mm,columnsep=20pt]{geometry} % Document margins
\usepackage[hang, small,labelfont=bf,up,textfont=it,up]{caption} % Custom captions under/above floats in tables or figures
\usepackage{booktabs} % Horizontal rules in tables

\usepackage{lettrine} % The lettrine is the first enlarged letter at the beginning of the text

\usepackage{enumitem} % Customized lists
\setlist[itemize]{noitemsep} % Make itemize lists more compact

\usepackage{abstract} % Allows abstract customization
\renewcommand{\abstractnamefont}{\normalfont\bfseries} % Set the "Abstract" text to bold
\renewcommand{\abstracttextfont}{\normalfont\small\itshape} % Set the abstract itself to small italic text

\usepackage{titlesec} % Allows customization of titles
\renewcommand\thesection{\Roman{section}} % Roman numerals for the sections
\renewcommand\thesubsection{\roman{subsection}} % roman numerals for subsections
\titleformat{\section}[block]{\large\scshape\centering}{\thesection.}{1em}{} % Change the look of the section titles
\titleformat{\subsection}[block]{\large}{\thesubsection.}{1em}{} % Change the look of the section titles

\usepackage{fancyhdr} % Headers and footers
\pagestyle{fancy} % All pages have headers and footers
\fancyhead{} % Blank out the default header
\fancyfoot{} % Blank out the default footer
\fancyhead[C]{Running title $\bullet$ May 2016 $\bullet$ Vol. XXI, No. 1} % Custom header text
\fancyfoot[RO,LE]{\thepage} % Custom footer text

\usepackage{titling} % Customizing the title section

\usepackage{hyperref} % For hyperlinks in the PDF

%----------------------------------------------------------------------------------------
%	TITLE SECTION
%----------------------------------------------------------------------------------------

\setlength{\droptitle}{-4\baselineskip} % Move the title up

\pretitle{\begin{center}\Huge\bfseries} % Article title formatting
\posttitle{\end{center}} % Article title closing formatting
\title{Immediate Changes in Assimilation Rates of Plant Leaves Damaged by Removal of an Area of Known Diameter } % Article title
\author{%
\textsc{Cory Andrew Hofstad}\thanks{A thank you or further information} \\[1ex] % Your name
\normalsize North Seattle College - UGR294  \\ % Your institution
\normalsize \href{mailto:cory.hofstad@seattlecolleges.edu}{cory.hofstad@seattlecolleges.edu} % Your email address
%\and % Uncomment if 2 authors are required, duplicate these 4 lines if more
%\textsc{Jane Smith}\thanks{Corresponding author} \\[1ex] % Second author's name
%\normalsize University of Utah \\ % Second author's institution
%\normalsize \href{mailto:jane@smith.com}{jane@smith.com} % Second author's email address
}
\date{\today} % Leave empty to omit a date
\renewcommand{\maketitlehookd}{%
\begin{abstract}
Research is being conducted by Undergraduate Students of North Seattle College in the fields of Ion Chromatography. In conducting this research, we learn to use the LICOR 6800 Ion Chromatography Device and collect data on plant assimilation via photosynthesis. As a research question involving Ion Chromatography, our group has decided to collect data from samples of local plant life before and after being damaged via herbivore feeding or from artificial methods of removing an area with a known diameter.
In readings conducted during this experiment, data shows that our sample plant shows high efficiency areas of assimilation when submitted to specific intensities of light from our Ion Chromatography Device. When submitted to artificial leaf damage in the form of a removal of a 6mm area via 1-hole punch, Ion Chromatography data shows that immediately after damage is inflicted, the measured section of leaf with a hole loses all ability to use light as a stimulant for photosynthesis.
The control data from this research can be used as a resource in maximizing efficiency of photosynthesis of our sample plant species using specific light intensities. Further research can be carried out in the field of plant damage to answer questions about recovery times, recovery methods and possible plant adaptation to damaged or missing sections with regard to photosynthesis efficiency.

\end{abstract}
}

%----------------------------------------------------------------------------------------

\begin{document}

% Print the title
\maketitle

%----------------------------------------------------------------------------------------
%	ARTICLE CONTENTS
%----------------------------------------------------------------------------------------

\section{Introduction}


\lettrine The plants in the wetland areas located on the north-east end of the North Seattle College Campus produce oxygen which can be measured by using Ion Chromatography. The assimilation of CO2 into oxygen can be measured for a variety of light levels, CO2 levels and humidity levels. We can then learn how individual plants react under various conditions and how assimilation is affected by these conditions.
Our team is doing research to determine if damage to a leaf effects the assimilation rates of the surrounding leaf surface on a dogwood plant within our local water table. Leaf damage can occur in situations in which herbivore predators damage sections of leaves, or when leaves are damaged by animals or breakage. The team has decided to use mealworms to test for natural damage and a standard one-hole punch used to remove a (magnitude) section of leaf.
Control readings are taken from a sample leaf without any previous leaf damage to determine a baseline of assimilation rates from a LICOR testing series which uses various light ranges to determine assimilation rates of a sample plant under normal conditions.
Tests need to be conducted on damaged leaves, so multiple leaf samples are subjected to herbivore trauma or the one-hole punch method of creating surface damage on a plant leaf.
The section of leaf is allowed time to react to the damage for 1 minute before Ion Chromatography testing is conducted using the LICOR system. Calculations are made to the LICOR system to account for the missing (damaged) leaf section.
The LICOR testing series is repeated on damaged leaf samples. Data from control readings and data from post damage readings can then be compared. Our team is then able to look for relations and/or patterns in Ion Chromatography data which provides readings on plant data including assimilation readings.
Our team is then able to compare data and come up with new questions


%------------------------------------------------

\section{Methods}

The study was conducted in the wetland area of North Seattle College (GPS HERE), on the north-east side of campus.  An attempt to take readings was made on Thursday October 19th, 2017 between 14:00-16:00, but testing was scrubbed after multiple equipment shutdowns due to excess condensation from heavy rain and weather conditions. Successful Ion Chromatography readings were taken on Tuesday October 24th, 2017 @ 14:43. These readings were taken during dry conditions which allowed for LICOR readings of plant data without equipment shutdowns during the 10+ minute readings intervals.
A group of plants was selected from a grouping of ground level plants with 10 – 20 leaves and average heights of 20cm - 45cm. A control reading was taken of a leaf with minimal pre-existing damage to collect data that is used to compare with data from our damaged samples to look for any changes in assimilation trends with and without damage. A leaf lowest to the ground was selected for Ion Chromatography readings because of its proximity to a resting area for our LICOR reading sensor.
Two Meal worms which were prepared for one week without food to promote herbivores damage to the leaves. Leaves were subjected to meal worm herbivores for a time period between 15 and 20 minutes in which they did not consume any of the sample leaves. During the time the leaves were subjected to herbivore stress, further research into meal worms was conducted in which we found that meal worms do not consume leaves, and instead consume a variety of other stock suck as carrots, wheat, oats, etc. Herbivore damage tests were concluded without success and the second method of using a standard 1-hole punch was used to artificially simulate herbivore stress on our plant samples.
A 1-hole punch with a diameter 6±0.5 mm was used to wound the leaf of a sample plant in order to gather data on leaf assimilation rates after damage from herbivores. A located of soft leaf material was removed from the leaf using the hole punch approximately 2/5 of the way from the stem to the tip of the leaf and 3/8 the distance from the center of the leaf to the outer edge. Major leaf veins were avoided to mimic damage found on other leaves of mostly soft tissue areas without heavy vein structure. The leaf was given one minute to adjust to the damage and allow the plant to exhibit any symptoms of trauma or shock which could be detected using LICOR data. 
Ion Chromatography readings were then taken from the section of leaf sample damaged by removal of a 6mm leaf section by method of one-hole punch. The LICOR device was programmed to account for the missing area of leaf. Our original leaf sample was damaged and accidentally severed from the plant after artificial damaged was induced by one-hole punch method. The weight of our LICOR device broke the leaf from the stem of the plant. Our group selected a leaf similar in size, ground height, lack of damage and plant size to carry out damage tests using the hole punch method to compare to our control data.
LICOR programming and readings were taken by instructor and overseer Ann Murkowski. Data from the LICOR 6800 was uploaded to Canvas for further analysis by the scientists. Group collaboration was carried out using Google Docs and Drive services which allowed the LICOR group to share and analyses the data while working on the Ion Chromatography Research. 
Ion Chromatography was used to test plant samples which we have for now identified as a dogwood plant.


\begin{itemize}
\item Donec dolor arcu, rutrum id molestie in, viverra sed diam
\item Curabitur feugiat
\item turpis sed auctor facilisis
\item arcu eros accumsan lorem, at posuere mi diam sit amet tortor
\item Fusce fermentum, mi sit amet euismod rutrum
\item sem lorem molestie diam, iaculis aliquet sapien tortor non nisi
\item Pellentesque bibendum pretium aliquet
\end{itemize}
\blindtext % Dummy text

Text requiring further explanation\footnote{Example footnote}.

%------------------------------------------------

\section{Results}

LICOR Ion Chromatography data shows two different trends in assimilation between our Control and Punch Experiments. Control data shows a highly variable range of assimilation rates over the light intensities which our LICOR device submitted our sample to.  Our punch data show a trendline with a constant negative slope of assimilation rate per increase in light intensity, lower overall efficiency and a much lower range of variance from the trend line of assimilation values for the domain of light intensities in which the plant was subjected to. 
Our control readings show the ability for our sample plant to produce a local maximum assimilation rate of .2 (µmol m⁻² s⁻¹) , when exposed to a 200 (µmol m⁻² s⁻¹) light intensity. The data recorded from the LICOR device shows an immediate loss in the plant’s ability utilize light in the process of assimilation through photosynthesis. Time of leaf puncture was at x = -60(s), the time our readings were started was at x = 0(s), Our test was concluded at x = 1500(s). 

\begin{tikzpicture} \begin{axis}[ height=9cm, width=9cm, grid=major, ] \addplot {-x^5 - 242}; \addlegendentry{model} \addplot coordinates { (-4.77778,2027.60977) (-3.55556,347.84069) (-2.33333,22.58953) (-1.11111,-493.50066) (0.11111,46.66082) (1.33333,-205.56286) (2.55556,-341.40638) (3.77778,-1169.24780) (5.00000,-3269.56775) }; \addlegendentry{estimate} \end{axis} \end{tikzpicture}

\blindtext % Dummy text

\begin{equation}
\label{eq:emc}
e = mc^2
\end{equation}

\blindtext % Dummy text

%------------------------------------------------

The equation of the line for our Ion Chromatography readings for Assimilation rates vs Light Intensity on our control sample was f(x) = (-3.98 e-5)(x) +( 0.106), our r2 value was = 0.071.
The equation of the line for our Ion Chromatography readings for assimilation rates vs light intensity on our punch sample was f(x) = (-5.38 e-5)(x)+(0.0775), our r2 value was = 0.807
Our data shows that the ability to photosynthesize in our sample leaf was non-existent immediately after hole punch damage has occurred. To further back our analysis of post damage data, readings could be taken of an area of the damaged leaf without the missing diameter. This would allow us to determine whether or not a missing circular diameter within our reading head contributed to our null assimilation rate readings.
While our data from the damaged leaf shows a lack of an ability to photosynthesize, our control data shows a rich variance in assimilation rates along the intensities of light in which our sample was submitted to. When data output from the LICOR 6800 device was used to create a graph of our control sample, it showed a maximum efficiency within specific light intensity ranges. This data suggests that plant assimilation can be artificially tuned via exposing a plant species to a specific intensity of light.


\section{Discussion}

Our control data shows that our plant samples show trends in efficiency within certain ranges of light intensity that would allow scientists to maximize the efficiency of plant assimilation in artificial conditions. Our Ion Chromatography data contained maximum assimilation rates at specific light intensities, that could be charted for different plant species in order to improve selection methods for producing larger volumes of oxygen for artificial environments and terraforming research.

\subsection{Subsection One}


A statement requiring citation \cite{Figueredo:2009dg}.
\blindtext % Dummy text

\subsection{Subsection Two}

\blindtext % Dummy text

%----------------------------------------------------------------------------------------
%	REFERENCE LIST
%----------------------------------------------------------------------------------------

\begin{thebibliography}{99} % Bibliography - this is intentionally simple in this template

\bibitemUsing The LI-6800 Portable Photosynthesis System Instruction Manual. (n.d.). Retrieved November 05, 2017, from https://www.licor.com/documents/6afbbpwybdanht6xrbgwicur4yohpx1n
\newblock LI-6800 Portable Photosynthesis System Instruction Manual
\newblock {\em Human Nature}, 20:317--330.
 
\end{thebibliography}

%----------------------------------------------------------------------------------------

\end{document}
